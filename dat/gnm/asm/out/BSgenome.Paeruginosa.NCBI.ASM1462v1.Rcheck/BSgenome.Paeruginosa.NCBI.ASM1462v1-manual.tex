\nonstopmode{}
\documentclass[letterpaper]{book}
\usepackage[times,inconsolata,hyper]{Rd}
\usepackage{makeidx}
\usepackage[utf8,latin1]{inputenc}
% \usepackage{graphicx} % @USE GRAPHICX@
\makeindex{}
\begin{document}
\chapter*{}
\begin{center}
{\textbf{\huge Package `BSgenome.Paeruginosa.NCBI.ASM1462v1'}}
\par\bigskip{\large \today}
\end{center}
\begin{description}
\raggedright{}
\item[Title]\AsIs{Full genome of Pseudomonas aeruginosa UCBPP-PA14}
\item[Description]\AsIs{Complete Genome}
\item[Version]\AsIs{1.0.0}
\item[Author]\AsIs{Dmytro Strunin}
\item[Maintainer]\AsIs{Bioconductor Package Maintainer }\email{maintainer@bioconductor.org}\AsIs{}
\item[Depends]\AsIs{BSgenome (>= 1.56.0)}
\item[Imports]\AsIs{BSgenome}
\item[Suggests]\AsIs{}
\item[License]\AsIs{Artistic-2.0}
\item[organism]\AsIs{Pseudomonas aeruginosa}
\item[common\_name]\AsIs{P. aeruginosa PA14}
\item[provider]\AsIs{NCBI}
\item[provider\_version]\AsIs{ASM1462v1}
\item[release\_date]\AsIs{2013-10-06}
\item[release\_name]\AsIs{Pseudomonas aeruginosa UCBPP-PA14 (g-proteobacteria)}
\item[source\_url]\AsIs{https://www.ncbi.nlm.nih.gov/assembly/GCF\_000014625.1}
\item[biocViews]\AsIs{AnnotationData, Genetics, BSgenome, Pseudomonas\_aeruginosa}
\item[NeedsCompilation]\AsIs{no}
\end{description}
\Rdcontents{\R{} topics documented:}
\inputencoding{utf8}
\HeaderA{BSgenome.Paeruginosa.NCBI.ASM1462v1}{Full genome of Pseudomonas aeruginosa UCBPP-PA14}{BSgenome.Paeruginosa.NCBI.ASM1462v1}
\aliasA{BSgenome.Paeruginosa.NCBI.ASM1462v1-package}{BSgenome.Paeruginosa.NCBI.ASM1462v1}{BSgenome.Paeruginosa.NCBI.ASM1462v1.Rdash.package}
\aliasA{Paeruginosa}{BSgenome.Paeruginosa.NCBI.ASM1462v1}{Paeruginosa}
\keyword{package}{BSgenome.Paeruginosa.NCBI.ASM1462v1}
\keyword{data}{BSgenome.Paeruginosa.NCBI.ASM1462v1}
%
\begin{Description}\relax
Complete Genome
\end{Description}
%
\begin{Note}\relax
This BSgenome data package was made from the following source data files:
\begin{alltt}
-- information not available --
  \end{alltt}


See \code{?\LinkA{BSgenomeForge}{BSgenomeForge}} and the BSgenomeForge
vignette (\code{vignette("BSgenomeForge")}) in the \pkg{BSgenome}
software package for how to make a BSgenome data package.
\end{Note}
%
\begin{Author}\relax
Dmytro Strunin
\end{Author}
%
\begin{SeeAlso}\relax
\begin{itemize}

\item{} \LinkA{BSgenome}{BSgenome} objects and the
\code{\LinkA{available.genomes}{available.genomes}} function
in the \pkg{BSgenome} software package.
\item{} \LinkA{DNAString}{DNAString} objects in the \pkg{Biostrings}
package.
\item{} The BSgenomeForge vignette (\code{vignette("BSgenomeForge")})
in the \pkg{BSgenome} software package for how to make a BSgenome
data package.

\end{itemize}

\end{SeeAlso}
%
\begin{Examples}
\begin{ExampleCode}
BSgenome.Paeruginosa.NCBI.ASM1462v1
genome <- BSgenome.Paeruginosa.NCBI.ASM1462v1
head(seqlengths(genome))


## ---------------------------------------------------------------------
## Genome-wide motif searching
## ---------------------------------------------------------------------
## See the GenomeSearching vignette in the BSgenome software
## package for some examples of genome-wide motif searching using
## Biostrings and the BSgenome data packages:
if (interactive())
    vignette("GenomeSearching", package="BSgenome")
\end{ExampleCode}
\end{Examples}
\printindex{}
\end{document}
